\documentclass{article}
\usepackage{amsmath,amsthm,amssymb}
\usepackage{graphicx}
\usepackage{cite}

\begin{document}
\title{Proyecto de Programación:\\ Licenciatura en Ciencia de la Computación.}
\author{Jorge Alejandro Echevarría Brunet.}
\date{}
\maketitle

% Logo y nombre de la Facultad.
\begin{figure}[h]
	% Logo
	\center
	\includegraphics[width=2cm]{images/matcom.jpg}
	\label{fig:logo}

	% Nombre
	\begin{center}
		\Large Facultad de Matemática y Computación.
	\end{center}
\end{figure}

\begin{abstract}
	Moogle! es una aplicación web *totalmente original* desarrollada con tecnología
	.NET Core 7.0, específicamente usando Blazor como *framework* para la interfaz
	gráfica, y en el lenguaje C\#, cuyo propósito es buscar inteligentemente un texto
	en un conjunto de documentos.
\end{abstract}

% Explicar que son los SRI, el Modelo Vectorial y el Modelo Vectorial Ponderado

\section*{Modelos de Recuperación de Información}
\subsection*{¿Qué es la Recuperación de Información?}
La recuperación de información es un proceso de comunicación. Es un medio por
el que los usuarios de un sistema o servicio de información pueden encontrar los documentos,
registros, imágenes gráficas, o registros de sonido que satisfagan sus necesidades o intereses.
\newline
\newline
Un Modelo de Recuperación de Información (MRI) es el responsable de encontrar información
relevante, la cual está presente en un registro documental que debe ser almacenado,
representado, analizado (manipulado) y mantenido.
\newpage
\textbf{Existen tres Modelos clásicos de Recuperación de Información}:
\begin{itemize}
	\item Booleano: Este modelo tradicionalmente utilizado en bases de datos establece una
	      relación booleana entre los términos y las consultas de búsqueda. Los documentos son vistos
	      como conjuntos de términos. La relevancia se basa en la coincidencia exacta entre los
	      términos de la consulta y los términos en los documentos.
	\item Vectorial: Este modelo representa los documentos y las consultas como vectores en
	      un espacio multidimensional, donde cada término corresponde a una dimensión. La relevancia
	      se determina mediante cálculos de similitud, como la similitud del coseno, entre los
	      vectores de los documentos y las consultas.
	\item Probabilístico: Estos modelos utilizan probabilidades para determinar la
	      relevancia de los documentos. El modelo de lenguaje de probabilidad, por ejemplo,
	      considera la probabilidad de que un documento genere una consulta o que una consulta
	      genere un documento, lo que permite calcular la probabilidad de relevancia.
\end{itemize}

\subsection*{¿Cuál fue modelo fue utilizado para el desarrollo de Moogle!?}
	El \textbf{Modelo de Espacio Vectorial Ponderado} es una variante del modelo vectorial, en la que se
	asignan pesos a los términos en función de su importancia relativa. Los pesos comúnmente
	utilizados son los basados en TF-IDF (frecuencia de término-inversa de frecuencia de documento),
	que tienen en cuenta tanto la frecuencia del término en el documento como en el conjunto de
	documentos.


\section{Arquitectura básica del Proyecto}
Durante el desarrollo de la aplicación fueron creadas las clases:
\begin{itemize}
	\item Reader
	\item TF-IDF
	\item Initialize
	\item Search
	\item Operators
	\item Snippet
\end{itemize}

% Página 2
\newpage
Así también como tuvo lugar la edición de otros archivos como fueron:
\begin{itemize}
	\item Program.cs
	\item Moogle.cs
	\item Index.razor
\end{itemize}

\section{Flujo de Datos}

\subsection{Preprocesamiento:}

Una vez iniciada la aplicación, se realiza un llamado al método \textbf{Feed} de la clase
\textbf{Initialize}, perteneciente al namespace \textbf{MoogleEngine} en la clase \textbf{Program}, la cual
pertenece al namespace \textbf{MoogleServer}. Este método
tiene como función principal hacer un llamado a todos los métodos de las clases
\textbf{Reader} y \textbf{TF-IDF}, que trabajan directamente con los diccionarios creados en la clase \textbf{Initialize}.
\newline
\newline

Los diccionarios a los que hacemos referencia son:
\begin{itemize}
	\item[-] Files
	\item[-] IDF
	\item[-] Texts
\end{itemize}
\textbf{Estructura interna de cada diccionario:}
\newline
\newline
- Files: Contiene todos los documentos de extensión “.txt’’, las palabras asociadas a cada
texto y el peso de los mismos. El cálculo de esta norma está dado por la siguiente ecuación
matemática:
\newline
\newline
$$
	W_{ij}=\frac{freq_{ij}}{max_{l}freq_{lj}} IDF_{i}
$$
\newline
\newline
Donde:
\begin{itemize}
	\item[.]$W_{ij}$: Representa el peso del término i en el documento j.
	\item[.]$freq_{ij}$: Representa la frecuencia del término i en el documento j.
	\item[.]$max_{l}freq_{lj}$: Representa el término l de mayor frecuencia en el documento j.
	\item[.]$IDF_{i}$: Representa la IDF del término i.
\end{itemize}

- IDF: Contiene todas las palabras de todos los textos y su Inverse Document Frequency (de ahí las siglas IDF)
asociada, la cual no es más que una norma para determinar la frecuencia inversa de un término
dado en todos los documentos donde aparezca. Esta norma se obtiene a partir de la
siguiente fórmula matemática:
\newline
$$
	IDF_{i}=\lg_{10}{(\frac{N}{n_{i}})}
$$
\newline
Donde:
\begin{itemize}
	\item [.]N: Representa la cantidad total de documentos.
	\item [.]$n_{i}$: Representa la cantidad de documentos donde ocurre el término i.
\end{itemize}

- Texts: Contiene cada documento de extensión “.txt” y asociado a cada uno su respectivo texto.



\subsection{Búsqueda:}
Al introducir la búsqueda (query) deseada y hacer click en el botón “Buscar”, en caso de ser una
búsqueda \textbf{NO} vacía, se aplicará una normalización a la misma, eliminando todos los
caracteres especiales no relevantes para el procesamiento, utilizando el método \textbf{Clean} de la
clase \textbf{Reader}. Luego se ubicará cada palabra normalizada de la query en el
diccionario \textbf{QueryWeight} mediante el método \textbf{FeedQueryWeight} (ambos pertenecientes a la clase
\textbf{Search}), donde a cada palabra se le asociará su peso. El cálculo del peso de cada término de la query
está dado por la siguiente fórmula:
$$
	W_{ij}=(\alpha +(1-\alpha )\frac{freq_{iq}}{max_{l}freq_{lq}})\lg_{10}{(\frac{N}{n_{i}})}
$$
Donde:
\begin{itemize}
	\item [.]$W_{ij}$: Representa el peso del término i en la query.
	\item [.]$freq_{iq}$: Representa la frecuencia del término i en la query.
	\item [.]$max_{l}freq_{lq}$: Representa el término l de mayor frecuencia en la query.
\end{itemize}
Luego se procede a buscar entre los documentos, aquellos que satisfagan total, o al
menos parcialmente la búsqueda.
Para ello se emplea la fórmula de Similitud de Cosenos:
$$
	Rsim(d_{j},q)=\frac{\sum_{i = 1}^{n} W_{ij} W_{iq}}
	{\sqrt{\sum_{i = 1}^{n} (W_{ij})^2} \sqrt{\sum_{i = 1}^{n} (W_{iq})^2}}
$$
\newpage
Donde:
\begin{itemize}
	\item[.] $Rsim(d_{j},q)$: Representa la puntuación (Score) del documento j con respecto a la query.
	\item[.] $W_{ij}$: Representa el peso del término i en el documento j.
	\item[.] $W_{iq}$: Representa el peso del término i en la query.
\end{itemize}

\section{Operadores}
Si en alguna palabra de nuestra búsqueda apareciera algún/unos de los operadores
	[$\sim $ ! $\wedge$ *], estos influirán directamente en el puntaje final de los documentos.
Haciendo uso del método \textbf{Clean} de la clase \textbf{Operators} se reconocerán dichos operadores y se
procederá de la siguiente forma en caso de existir algún/unos de ellos:
\begin{itemize}
	\item \textbf{!} Operador de \textbf{NO Aparición}: Si alguna palabra presenta este operador, todo documento que
	      la contenga será automáticamente descartado de los resultados finales.
	\item \textbf{$\wedge$} Operador de \textbf{Solo Aparición}: Si alguna palabra presenta este operador, todo documento que no
	      la contenga será automáticamente descartado de los resultados finales.
	\item \textbf{*} Operador de \textbf{Importancia}: Si alguna palabra presenta este operador, todo documento que
	      la contenga adquirirá mayor relevancia en dependencia de la cantidad de operadores ‘*’ que la
	      precedan, para llevar este conteo se utiliza el método \textbf{Charcounter} de la clase \textbf{Operators}.
	\item \textbf{$\sim $} Operador de \textbf{Cercanía}: Si la palabra que lo posee es la primera de
	      la búsqueda, este será ignorado puesto que se tienen en cuenta tanto la palabra
	      precedida del operador de cercanía, como su antecesora. En cualquier otro caso se
	      determinará la distancia entre ambas palabras mediante el método \textbf{Distance} de la
	      clase \textbf{Operators}. De ser esta menor o igual a 10 palabras, su puntaje será
	      considerablemente alto, de no ser así, se tendrá en cuenta al mostrar los
	      resultados finales.
\end{itemize}

\newpage

\begin{figure}

	\section{Resultados}
	Una vez ubicados los 3 documentos de mayor puntaje, estos serán mostrados en pantalla en orden de
	relevancia descendente, precedidos con una línea de texto donde aparezca la
	palabra introducida en nuestra búsqueda.
	Para ello se utilizó el método \textbf{Show} de la clase \textbf{Snippet}.\\
	En caso de realizar una búsqueda donde las coincidencias sean nulas, se ofrecerá en pantalla
	una sugerencia de búsqueda, la cual será la palabra más parecida a la primera palabra de
	nuestra búsqueda. Para ello se utilizaron también los métodos de la clase \textbf{Snippet}:
	\begin{itemize}
		\item[1-] \textbf{EditDistance}: Función basada en el algoritmo conocido como ”Distancia de
			Levenshtein”, el cual determina la mínima cantidad de ediciones a realizar para convertir
			una cadena de caracteres en otra.
		\item[2-] \textbf{Suggestion}: Función que determina en base a los resultados obtenidos
			al aplicar el método \textbf{EditDistance}, la palabra con mayor similitud a la primera
			palabra de nuestra búsqueda.
	\end{itemize}

	\section*{Manual de Usuario:}
	\begin{flushleft}
		Al iniciar Moogle! y acceder a la página se nos muestra como en la siguiente imagen:
	\end{flushleft}
	\begin{center}
		\includegraphics*[width=10cm]{images/start.jpg}
	\end{center}
\end{figure}

\begin{figure}
	\section*{¿Cómo realizar una búsqueda con Moogle!?:}
	\begin{itemize}
		\item[1-] Escribir en la caja de texto la búsqueda a realizar.
		\item[2-] Hacer click en el botón Buscar.
	\end{itemize}

	\includegraphics*[width=16.5cm]{images/search.jpg}
	\begin{flushleft}
		En caso de no encontrar ningún resultado para nuestra búsqueda, podemos hacer click
		en la sugerencia que muestra la aplicación y así obtendremos un documento que podría
		estar relacionado con nuestra búsqueda.

	\end{flushleft}
\end{figure}

\end{document}