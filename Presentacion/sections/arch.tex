\section{Arquitectura Básica}
\begin{frame}{Arquitectura Básica}
	Durante el desarrollo de la aplicación fueron creadas las clases:
	\begin{itemize}
		\item Reader: Identifica los documentos y normaliza sus textos correspondientes.
		\item TF-IDF: Calcula el peso de los documentos previamente identificados.
		\item Search: Normaliza la búsqueda (query) introducida y determina los documentos más relevantes.
		\item Operators: Trabaja con los operadores utilizados en la query, influyendo directamente
		      en el puntaje de los documentos con más relevancia.
		\item Snippet: Muestra una porción de texto de cada documento relevante y una posible
		      sugerencia al usuario.
		\item Initialize: Utiliza las funciones de las clases \textbf{Reader} y \textbf{TF-IDF}
		      a la hora de lanzar la aplicación.
	\end{itemize}
\end{frame}
% Objetos mas importantes de las clases que son los diccionarios Files IDF Texts
\begin{frame}{}
La estructura de datos más importante utilizada son los diccionarios.\\
Haciendo uso de estos se realizan las operaciones pertinentes para determinar los mejores
resultados de acuerdo a la consulta realizada.
\newline
\newline
Tengamos en cuenta que los principales diccionarios a utilizar durante el Preprocesamiento son:
\begin{itemize}
	\item Files
	\item IDF
\end{itemize}
Mientras que durante la Consulta:
\begin{itemize}
	\item QueryWeight
	\item Score
\end{itemize}
% 	\item QueryWeight
% 	\item Score
% \end{itemize}
\end{frame}

